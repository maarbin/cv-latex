\documentclass[10pt, letterpaper]{article}

% Packages:
\usepackage[
    ignoreheadfoot, % set margins without considering header and footer
    top=2 cm, % seperation between body and page edge from the top
    bottom=2 cm, % seperation between body and page edge from the bottom
    left=2 cm, % seperation between body and page edge from the left
    right=2 cm, % seperation between body and page edge from the right
    footskip=1.0 cm, % seperation between body and footer
    % showframe % for debugging 
]{geometry} % for adjusting page geometry
\usepackage{titlesec} % for customizing section titles
\usepackage{tabularx} % for making tables with fixed width columns
\usepackage{array} % tabularx requires this
\usepackage[dvipsnames]{xcolor} % for coloring text
\definecolor{primaryColor}{RGB}{0, 79, 144} % define primary color
\usepackage{enumitem} % for customizing lists
\usepackage{fontawesome5} % for using icons
\usepackage{amsmath} % for math
\usepackage[
    pdftitle={Marta Bińkowska CV},
    pdfauthor={Marta Bińkowska},
    pdfcreator={My project},
    colorlinks=true,
    urlcolor=primaryColor
]{hyperref} % for links, metadata and bookmarks
\usepackage[pscoord]{eso-pic} % for floating text on the page
\usepackage{calc} % for calculating lengths
\usepackage{bookmark} % for bookmarks
\usepackage{lastpage} % for getting the total number of pages
\usepackage{changepage} % for one column entries (adjustwidth environment)
\usepackage{paracol} % for two and three column entries
\usepackage{ifthen} % for conditional statements
\usepackage{needspace} % for avoiding page brake right after the section title
\usepackage{iftex} % check if engine is pdflatex, xetex or luatex

% Ensure that generate pdf is machine readable/ATS parsable:
\ifPDFTeX
    \input{glyphtounicode}
    \pdfgentounicode=1
    \usepackage[T1]{fontenc} % this breaks sb2nov
    \usepackage[utf8]{inputenc}
    \usepackage[polish,english]{babel} % for language support
    \usepackage{lmodern}
\fi



% Some settings:
\AtBeginEnvironment{adjustwidth}{\partopsep0pt} % remove space before adjustwidth environment
\pagestyle{empty} % no header or footer
\setcounter{secnumdepth}{0} % no section numbering
\setlength{\parindent}{0pt} % no indentation
\setlength{\topskip}{0pt} % no top skip
\setlength{\columnsep}{0cm} % set column seperation
\makeatletter
\let\ps@customFooterStyle\ps@plain % Copy the plain page style


\patchcmd{\ps@customFooterStyle}{\thepage}{
  \parbox{\textwidth}{
    \centering
    \color{gray}\footnotesize
    Wyrażam zgodę na przetwarzanie moich danych osobowych zawartych w CV dla potrzeb niezbędnych do realizacji procesu rekrutacji zgodnie z art. 6 ust. 1 lit. a RODO (Rozporządzenie Parlamentu Europejskiego i Rady (UE) 2016/679 z dnia 27 kwietnia 2016 r.).

  }
}{}{}


\makeatother
\pagestyle{customFooterStyle}

\titleformat{\section}{\needspace{4\baselineskip}\bfseries\large}{}{0pt}{}[\vspace{1pt}\titlerule]

\titlespacing{\section}{
    % left space:
    -1pt
}{
    % top space:
    0.3 cm
}{
    % bottom space:
    0.2 cm
} % section title spacing

\renewcommand\labelitemi{$\circ$} % custom bullet points
\newenvironment{highlights}{
    \begin{itemize}[
        topsep=0.10 cm,
        parsep=0.10 cm,
        partopsep=0pt,
        itemsep=0pt,
        leftmargin=0.4 cm + 10pt
    ]
}{
    \end{itemize}
} % new environment for highlights

\newenvironment{highlightsforbulletentries}{
    \begin{itemize}[
        topsep=0.10 cm,
        parsep=0.10 cm,
        partopsep=0pt,
        itemsep=0pt,
        leftmargin=10pt
    ]
}{
    \end{itemize}
} % new environment for highlights for bullet entries


\newenvironment{onecolentry}{
    \begin{adjustwidth}{
        0.2 cm + 0.00001 cm
    }{
        0.2 cm + 0.00001 cm
    }
}{
    \end{adjustwidth}
} % new environment for one column entries

\newenvironment{twocolentry}[2][]{
    \onecolentry
    \def\secondColumn{#2}
    \setcolumnwidth{\fill, 4.5 cm}
    \begin{paracol}{2}
}{
    \switchcolumn \raggedleft \secondColumn
    \end{paracol}
    \endonecolentry
} % new environment for two column entries

\newenvironment{header}{
    \setlength{\topsep}{0pt}\par\kern\topsep\centering\linespread{1.5}
}{
    \par\kern\topsep
} % new environment for the header

\newcommand{\placelastupdatedtext}{% \placetextbox{<horizontal pos>}{<vertical pos>}{<stuff>}
  \AddToShipoutPictureFG*{% Add <stuff> to current page foreground
    \put(
        \LenToUnit{\paperwidth-2 cm-0.2 cm+0.05cm},
        \LenToUnit{\paperheight-1.0 cm}
    ){\vtop{{\null}\makebox[0pt][c]{
        \small\color{gray}\textit{}\hspace{\widthof{}}
    }}}%
  }%
}%

% save the original href command in a new command:
\let\hrefWithoutArrow\href

% new command for external links:
\renewcommand{\href}[2]{\hrefWithoutArrow{#1}{\ifthenelse{\equal{#2}{}}{ }{#2 }\raisebox{.15ex}{\footnotesize \faExternalLink*}}}


\begin{document}
    \newcommand{\AND}{\unskip
        \cleaders\copy\ANDbox\hskip\wd\ANDbox
        \ignorespaces
    }
    \newsavebox\ANDbox
    \sbox\ANDbox{}

    \placelastupdatedtext
    \begin{header}
        \textbf{\fontsize{24 pt}{24 pt}\selectfont Marta Bińkowska}

        \vspace{0.3 cm}

        \normalsize
        \mbox{{\color{black}\footnotesize\faMapMarker*}\hspace*{0.13cm}Poland}%
        \kern 0.25 cm%
        \AND%
        \kern 0.25 cm%
        \mbox{\hrefWithoutArrow{mailto:mail@xyz.com}{\color{black}{\footnotesize\faEnvelope[regular]}\hspace*{0.13cm}mail@xyz.com}}%
        \kern 0.25 cm%
        \AND%
        \kern 0.25 cm%
        \mbox{\hrefWithoutArrow{tel:123 456 789}{\color{black}{\footnotesize\faPhone*}\hspace*{0.13cm}123 456 789}}%
        \kern 0.25 cm%
        %\AND%
        %\kern 0.25 cm%
       % \mbox{\hrefWithoutArrow{https://yourwebsite.com/}{\color{black}{\footnotesize\faLink}\hspace*{0.13cm}yourwebsite.com}}%
       % \kern 0.25 cm%
        \AND%
        \kern 0.25 cm%
        \mbox{\hrefWithoutArrow{https://linkedin.com/in/binkowska-marta/}{\color{black}{\footnotesize\faLinkedinIn}\hspace*{0.13cm}binkowska-marta}}%
        \kern 0.25 cm%
        \AND%
        \kern 0.25 cm%
        \mbox{\hrefWithoutArrow{https://github.com/maarbin}{\color{black}{\footnotesize\faGithub}\hspace*{0.13cm}maarbin}}%
    \end{header}

    \vspace{0.3 cm - 0.3 cm}


    \section{O mnie}



        
        \begin{onecolentry}
            Jestem samodzielną specjalistką z mocnym nastawieniem na rozwiązywanie problemów, co przekłada się na proaktywne i kreatywne podejście do codziennej pracy. Aby pokazać to podejście już na poziomie CV, zaprojektowałam ten dokument w LaTeX i zautomatyzowałam generowanie wersji językowych oraz RODO za pomocą Pythona i silnika szablonów Jinja2. Kod źródłowy można znaleźć w moim repozytorium na GitHubie, gdzie dzielę się także innymi projektami, takimi jak model i dashboard w Power BI. Poza ciągłym rozwojem kompetencji wierzę też w dzielenie się wiedzą w zespole - prowadziłam szkolenia wewnętrzne na każdym stanowisku, które zajmowałam. Obecnie szukam nowych możliwości które pozwolą mi rozwijać się, jednocześnie wnosząc wartość do projektów i zespołu.

        \end{onecolentry}




      \section{Wykształcenie}



        
      
      \begin{twocolentry}{
          \textit{2020~–~2022}
      }
          \textbf{Uniwersytet Szczeciński}
      
          \textit{Informatyka i Ekonometria (Magister)}
      \end{twocolentry}
      \vspace{0.15 cm}
      
      \begin{twocolentry}{
          \textit{2017~–~2020}
      }
          \textbf{Uniwersytet Szczeciński}
      
          \textit{Finanse i Rachunkowość (Licencjat)}
      \end{twocolentry}
      \vspace{0.15 cm}
      



    
      \section{Doświadczenie}



        
      
      \begin{twocolentry}{
          \textit{Szczecin, Polska}
      
          \textit{Listopad 2022~–~Obecnie}
      }
          \textbf{Data Specialist}
      
          \textit{Macrobond Financial}
      \end{twocolentry}
      
      \vspace{0.10 cm}
      \begin{onecolentry}
          \begin{highlights}
              
              \item Aktualizacja i zarządzanie danymi ekonomicznymi oraz finansowymi; zapewnianie wysokiej jakości danych
              
              \item Czyszczenie zbiorów danych i korygowanie niespójności lub anomalii
              
              \item Identyfikacja i wdrażanie automatyzacji w procesach przetwarzania danych
              
          \end{highlights}
      \end{onecolentry}
      
      \vspace{0.2 cm}
      
      \begin{twocolentry}{
          \textit{Szczecin, Polska}
      
          \textit{Lipiec~2019~–~Październik~2022}
      }
          \textbf{Księgowa}
      
          \textit{Demant Business Services}
      \end{twocolentry}
      
      \vspace{0.10 cm}
      \begin{onecolentry}
          \begin{highlights}
              
              \item Tworzenie cyklicznych raportów i zarządzanie dużymi wolumenami danych w Excelu
              
              \item Uzgadnianie sald i rozliczeń międzyokresowych
              
              \item Udział w projektach optymalizacji procesów księgowych
              
          \end{highlights}
      \end{onecolentry}
      
      \vspace{0.2 cm}
      




    
    \section{Certyfikaty}


        
    
    \begin{samepage}
        \begin{twocolentry}{
            Grudzień~2023
        }
            \textbf{PCEP [30-01]– Certified Entry-Level Python Programmer}
    
            \vspace{0.10 cm}
    
            \mbox{OpenEDG Python Institute}
        \end{twocolentry}
    
        \vspace{0.10 cm}
    
        \begin{onecolentry}
            \href{https://verify.openedg.org/?id=kBAL.rJzU.gLN2}{Certificate ID: }
        \end{onecolentry}
    \end{samepage}
    
    \vspace{0.15 cm}
    


    
    \section{Umiejętności}


    \begin{itemize}
        
        \item \textbf{Języki }: Polski (ojczysty), Angielski (C1) 
        
      
        
        \item \textbf{Programowanie }: SQL (średniozaawansowany), Python (średniozaawansowany) 
        
      
        
        \item \textbf{Dane }: Excel (zaawansowany), Power BI (średniozaawansowany), Power Query 
        
      
        
        \item \textbf{Narzędzia }: Git, SVN, Zsh 
        
      \end{itemize}


    

\end{document}