\documentclass[10pt, letterpaper]{article}

% Packages:
\usepackage[
    ignoreheadfoot, % set margins without considering header and footer
    top=2 cm, % seperation between body and page edge from the top
    bottom=2 cm, % seperation between body and page edge from the bottom
    left=2 cm, % seperation between body and page edge from the left
    right=2 cm, % seperation between body and page edge from the right
    footskip=1.0 cm, % seperation between body and footer
    % showframe % for debugging 
]{geometry} % for adjusting page geometry
\usepackage{titlesec} % for customizing section titles
\usepackage{tabularx} % for making tables with fixed width columns
\usepackage{array} % tabularx requires this
\usepackage[dvipsnames]{xcolor} % for coloring text
\definecolor{primaryColor}{RGB}{0, 79, 144} % define primary color
\usepackage{enumitem} % for customizing lists
\usepackage{fontawesome5} % for using icons
\usepackage{amsmath} % for math
\usepackage[
    pdftitle={Marta Bińkowska CV},
    pdfauthor={Marta Bińkowska},
    pdfcreator={My project},
    colorlinks=true,
    urlcolor=primaryColor
]{hyperref} % for links, metadata and bookmarks
\usepackage[pscoord]{eso-pic} % for floating text on the page
\usepackage{calc} % for calculating lengths
\usepackage{bookmark} % for bookmarks
\usepackage{lastpage} % for getting the total number of pages
\usepackage{changepage} % for one column entries (adjustwidth environment)
\usepackage{paracol} % for two and three column entries
\usepackage{ifthen} % for conditional statements
\usepackage{needspace} % for avoiding page brake right after the section title
\usepackage{iftex} % check if engine is pdflatex, xetex or luatex

% Ensure that generate pdf is machine readable/ATS parsable:
\ifPDFTeX
    \input{glyphtounicode}
    \pdfgentounicode=1
    \usepackage[T1]{fontenc} % this breaks sb2nov
    \usepackage[utf8]{inputenc}
    \usepackage[polish,english]{babel} % for language support
    \usepackage{lmodern}
\fi



% Some settings:
\AtBeginEnvironment{adjustwidth}{\partopsep0pt} % remove space before adjustwidth environment
\pagestyle{empty} % no header or footer
\setcounter{secnumdepth}{0} % no section numbering
\setlength{\parindent}{0pt} % no indentation
\setlength{\topskip}{0pt} % no top skip
\setlength{\columnsep}{0cm} % set column seperation
\makeatletter
\let\ps@customFooterStyle\ps@plain % Copy the plain page style


\patchcmd{\ps@customFooterStyle}{\thepage}{
  \parbox{\textwidth}{
    \centering
    \color{gray}\footnotesize
    I hereby give consent for my personal data to be processed for the purposes of the current recruitment process,  in accordance with Article 6(1)(a) of the GDPR (Regulation (EU) 2016/679 of the European Parliament and of the Council of 27 April 2016).

  }
}{}{}


\makeatother
\pagestyle{customFooterStyle}

\titleformat{\section}{\needspace{4\baselineskip}\bfseries\large}{}{0pt}{}[\vspace{1pt}\titlerule]

\titlespacing{\section}{
    % left space:
    -1pt
}{
    % top space:
    0.3 cm
}{
    % bottom space:
    0.2 cm
} % section title spacing

\renewcommand\labelitemi{$\circ$} % custom bullet points
\newenvironment{highlights}{
    \begin{itemize}[
        topsep=0.10 cm,
        parsep=0.10 cm,
        partopsep=0pt,
        itemsep=0pt,
        leftmargin=0.4 cm + 10pt
    ]
}{
    \end{itemize}
} % new environment for highlights

\newenvironment{highlightsforbulletentries}{
    \begin{itemize}[
        topsep=0.10 cm,
        parsep=0.10 cm,
        partopsep=0pt,
        itemsep=0pt,
        leftmargin=10pt
    ]
}{
    \end{itemize}
} % new environment for highlights for bullet entries


\newenvironment{onecolentry}{
    \begin{adjustwidth}{
        0.2 cm + 0.00001 cm
    }{
        0.2 cm + 0.00001 cm
    }
}{
    \end{adjustwidth}
} % new environment for one column entries

\newenvironment{twocolentry}[2][]{
    \onecolentry
    \def\secondColumn{#2}
    \setcolumnwidth{\fill, 4.5 cm}
    \begin{paracol}{2}
}{
    \switchcolumn \raggedleft \secondColumn
    \end{paracol}
    \endonecolentry
} % new environment for two column entries

\newenvironment{header}{
    \setlength{\topsep}{0pt}\par\kern\topsep\centering\linespread{1.5}
}{
    \par\kern\topsep
} % new environment for the header

\newcommand{\placelastupdatedtext}{% \placetextbox{<horizontal pos>}{<vertical pos>}{<stuff>}
  \AddToShipoutPictureFG*{% Add <stuff> to current page foreground
    \put(
        \LenToUnit{\paperwidth-2 cm-0.2 cm+0.05cm},
        \LenToUnit{\paperheight-1.0 cm}
    ){\vtop{{\null}\makebox[0pt][c]{
        \small\color{gray}\textit{}\hspace{\widthof{}}
    }}}%
  }%
}%

% save the original href command in a new command:
\let\hrefWithoutArrow\href

% new command for external links:
\renewcommand{\href}[2]{\hrefWithoutArrow{#1}{\ifthenelse{\equal{#2}{}}{ }{#2 }\raisebox{.15ex}{\footnotesize \faExternalLink*}}}


\begin{document}
    \newcommand{\AND}{\unskip
        \cleaders\copy\ANDbox\hskip\wd\ANDbox
        \ignorespaces
    }
    \newsavebox\ANDbox
    \sbox\ANDbox{}

    \placelastupdatedtext
    \begin{header}
        \textbf{\fontsize{24 pt}{24 pt}\selectfont Marta Bińkowska}

        \vspace{0.3 cm}

        \normalsize
        \mbox{{\color{black}\footnotesize\faMapMarker*}\hspace*{0.13cm}Poland}%
        \kern 0.25 cm%
        \AND%
        \kern 0.25 cm%
        \mbox{\hrefWithoutArrow{mailto:mail@xyz.com}{\color{black}{\footnotesize\faEnvelope[regular]}\hspace*{0.13cm}mail@xyz.com}}%
        \kern 0.25 cm%
        \AND%
        \kern 0.25 cm%
        \mbox{\hrefWithoutArrow{tel:123 456 789}{\color{black}{\footnotesize\faPhone*}\hspace*{0.13cm}123 456 789}}%
        \kern 0.25 cm%
        %\AND%
        %\kern 0.25 cm%
       % \mbox{\hrefWithoutArrow{https://yourwebsite.com/}{\color{black}{\footnotesize\faLink}\hspace*{0.13cm}yourwebsite.com}}%
       % \kern 0.25 cm%
        \AND%
        \kern 0.25 cm%
        \mbox{\hrefWithoutArrow{https://linkedin.com/in/binkowska-marta/}{\color{black}{\footnotesize\faLinkedinIn}\hspace*{0.13cm}binkowska-marta}}%
        \kern 0.25 cm%
        \AND%
        \kern 0.25 cm%
        \mbox{\hrefWithoutArrow{https://github.com/maarbin}{\color{black}{\footnotesize\faGithub}\hspace*{0.13cm}maarbin}}%
    \end{header}

    \vspace{0.3 cm - 0.3 cm}


    \section{Summary}



        
        \begin{onecolentry}
            I am a self-driven professional with a strong focus on problem-solving, which translates into a proactive and creative approach to everyday work. As a way to reflect this mindset already at the CV stage, I designed this document using LaTeX and automated the generation of different language and GDPR versions with Python and the Jinja2 templating engine. The source code is available in my GitHub repository, which also features other projects, including an end-to-end analytical project developed with Python and Power BI. In addition to continuously developing my skills, I also believe in knowledge sharing within the team. I have conducted internal trainings in every position I have held. I’m currently looking for opportunities where I can grow while contributing to meaningful projects and team development.

        \end{onecolentry}




      \section{Education}



        
      
      \begin{twocolentry}{
          \textit{2020~–~2022}
      }
          \textbf{University of Szczecin}
      
          \textit{MS in Econometrics and IT Applications}
      \end{twocolentry}
      \vspace{0.15 cm}
      
      \begin{twocolentry}{
          \textit{2017~–~2020}
      }
          \textbf{University of Szczecin}
      
          \textit{BS in Finance and Accounting}
      \end{twocolentry}
      \vspace{0.15 cm}
      



    
      \section{Experience}



        
      
      \begin{twocolentry}{
          \textit{Szczecin, Poland}
      
          \textit{November~2022~–~Present}
      }
          \textbf{Data Specialist}
      
          \textit{Macrobond Financial}
      \end{twocolentry}
      
      \vspace{0.10 cm}
      \begin{onecolentry}
          \begin{highlights}
              
              \item Updating and managing economic and financial data on the platform; ensuring high data quality
              
              \item Cleaning datasets and correcting data inconsistencies or anomalies
              
              \item Identifying and implementing automation in data processing workflows
              
          \end{highlights}
      \end{onecolentry}
      
      \vspace{0.2 cm}
      
      \begin{twocolentry}{
          \textit{Szczecin, Poland}
      
          \textit{July~2019~–~October 2022}
      }
          \textbf{Accountant}
      
          \textit{Demant Business Services}
      \end{twocolentry}
      
      \vspace{0.10 cm}
      \begin{onecolentry}
          \begin{highlights}
              
              \item Creating periodic reports and managing large data volumes in Excel
              
              \item Reconciling balances and accruals
              
              \item Participating in accounting process optimization projects
              
          \end{highlights}
      \end{onecolentry}
      
      \vspace{0.2 cm}
      




    
    \section{Certifications}


        
    
    \begin{samepage}
        \begin{twocolentry}{
            December~2023
        }
            \textbf{PCEP [30-01]– Certified Entry-Level Python Programmer}
    
            \vspace{0.10 cm}
    
            \mbox{OpenEDG Python Institute}
        \end{twocolentry}
    
        \vspace{0.10 cm}
    
        \begin{onecolentry}
            \href{https://verify.openedg.org/?id=kBAL.rJzU.gLN2}{Certificate ID: kBAL.rJzU.gLN2}
        \end{onecolentry}
    \end{samepage}
    
    \vspace{0.15 cm}
    


    
    \section{Skills}


    \begin{itemize}
        
        \item \textbf{Languages }: Polish (native), English (C1) 
        
      
        
        \item \textbf{Programming }: SQL (intermediate), Python (intermediate) 
        
      
        
        \item \textbf{Data }: Excel (advanced), Power BI (intermediate), Power Query 
        
      
        
        \item \textbf{Tools }: Git, SVN, Zsh 
        
      \end{itemize}


    

\end{document}